% coding:utf-8

\section{Inbetriebnahme}
Nach der Bestückung werden die Platinen in Betrieb genommen. Zusätzlich wird 
der zulässige Bereich für den Widerstand des Kabels geprüft. Dazu wird die 
Platine mit einer Betriebsspannung von $5 V$ und einer Akkuspannung von $6.6 V$ 
gespiesen. Anschliessend werden Widerstände gemäss nachfolgender Tabelle an die 
Tasteranschlüsse verbunden und die Ausgangsspannung gemessen

\subsection{Inbetriebnahme Platine 1}
\begin{tabular}{@{}lllll}
                & \multicolumn{2}{c}{$U_a$ Ja}  & \multicolumn{2}{c}{$U_a$ Nein} \\
  $R$           & Erwartet  & Gemessen  & Erwartet  & Gemessen  \\
  offen         & $\geq3V$  & $4.88V$   & $\geq3V$  & $4.88V$   \\
  $0 \Omega$    & $\leq1.5V$& $67.8mV$  & $\leq1.5V$& $67.5mV$  \\
  $10 \Omega$   & $\leq1.5V$& $67.9mV$  & $\leq1.5V$& $67.6mV$  \\
  $100 \Omega$  & $\leq1.5V$& $67.3mV$  & $\leq1.5V$& $67.0mV$  \\
  $1 k\Omega$   & $\leq1.5V$& $75.8mV$  & $\leq1.5V$& $76.4mV$  \\
  $10 k\Omega$  & $\geq3V$  & $4.88V$   & $\geq3V$  & $4.88V$   \\
\end{tabular}

\subsection{Inbetriebnahme Platine 2}
\begin{tabular}{@{}lllll}
                & \multicolumn{2}{c}{$U_a$ Ja}  & \multicolumn{2}{c}{$U_a$ Nein} \\
  $R$           & Erwartet  & Gemessen  & Erwartet  & Gemessen  \\
  offen         & $\geq3V$  & $4.88V$   & $\geq3V$  & $4.88V$   \\
  $0 \Omega$    & $\leq1.5V$& $68.6mV$  & $\leq1.5V$& $69.1mV$  \\
  $10 \Omega$   & $\leq1.5V$& $68.5mV$  & $\leq1.5V$& $69.3mV$  \\
  $100 \Omega$  & $\leq1.5V$& $67.8mV$  & $\leq1.5V$& $68.3mV$  \\
  $1 k\Omega$   & $\leq1.5V$& $76.6mV$  & $\leq1.5V$& $76.8mV$  \\
  $10 k\Omega$  & $\geq3V$  & $4.88V$   & $\geq3V$  & $4.88V$   \\
\end{tabular}